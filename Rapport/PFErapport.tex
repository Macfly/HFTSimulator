%Rapport PFE
\documentclass[a4paper,10pt]{article}
\usepackage{numprint}
\usepackage[squaren, Gray, cdot]{SIunits}
\usepackage{slashbox}
\usepackage[utf8]{inputenc}
\usepackage[T1]{fontenc}     
\usepackage[francais]{babel}
\usepackage{amsmath}
\usepackage{listings} 
\usepackage{amssymb}
\usepackage{fullpage}
\usepackage{multirow}
%\usepackage{graphicx}
\usepackage[pdftex]{graphicx}
\usepackage{textcomp}
\usepackage{amsthm}
 \usepackage{dsfont}
%\renewcommand{\appendixtocname}{Annexes} 
%\renewcommand{\appendixpagename}{Annexes} 
\begin{document}

\begin{titlepage}

\begin{figure}[h!t]



\begin{minipage}[c]{.46\linewidth}
      \includegraphics[width = 30mm]{D:/docs/StageAAAiC/Unice.jpeg}
   \end{minipage} \hfill 
   \begin{minipage}[c]{.30\linewidth}
      \includegraphics[width = 40mm]{D:/docs/StageAAAiC/logoP.jpg}
   \end{minipage}
   
\end{figure}
\vspace{5cm}

\begin{center}

\LARGE{{Projet de fin d'études\\
Simulateur de marché financier et Trading Haute Fréquence} }
\vspace{0.5cm}

\vspace{0.5cm}
\large{{Maxime Bonelli \quad Youness Abdessadek\\Corentin Valleroy \quad Jérémie Montiel} \\
\vspace{2cm}
   {Encadrant : Dr Mireille Bossy\\
   }
   
 \vspace{0.5cm}

   
   \begin{center}


\vspace{0.5cm}
 {Année scolaire 2012/2013}

\end{center}
}\end{center}

\end{titlepage}


\vspace*{\stretch{1}}
\begin{abstract}


Le but de ce rapport est de mettre en évidence les résultats obtenus durant le projet de fin d'études effectuée dans le cadre du master IMAFA à l'école Polytech Nice-Sophia. Les travaux, encadrés par le Dr Mireille Bossy, consistaient à implémenter des stratégies de trading haute fréquence (HFT) sur un simulateur de marché financier, à l’aide du langage de programmation C++.
\\		

Ce document se compose de trois parties principales. En premier lieu l'organisation du projet est présentée. Le modèle utilisé pour simuler le marché est ensuite exposé. Enfin, les stratégies de HFT implémentées sont l'objet de la dernière partie.
\vspace*{\stretch{1}}
\end{abstract}

\newpage
\tableofcontents 
\newpage
\section{Organisation du projet}


%\begin{center}
%\label{gaussienne3}
%\begin{figure}[!h]
%\centering \includegraphics[width = 113mm]
%{D:/docs/MAM5/MCTanre/lambda.png}
%\caption{\label{lambda} Écarts-types des estimateurs en fonction de $\lambda$}
%\end{figure}
%\end{center}
%
%
%\begin{figure}[!h]
%\begin{minipage}[c]{.20\linewidth}
%      \includegraphics[width = 98mm]{D:/docs/MAM5/MCTanre/XX.png}
%   \end{minipage} \hfill 
%   \begin{minipage}[c]{.43\linewidth}
%      \includegraphics[width = 98mm]{D:/docs/MAM5/MCTanre/YY.png}
%   \end{minipage}
%   \caption{\label{XY} Trajectoires des processus $X$ et $Y$}
%\end{figure} 



\clearpage
\newpage
\addcontentsline{toc}{section}{\numberline{}{Conclusion}}
\section*{Conclusion}

Ce projet constituait une véritable application des notions et des outils étudiés durant le cours "Méthodes numériques probabilistes".
Le cœur du projet était d'étudier les différentes méthodes de simulations Monte-Carlo et les propriétés des résultats qu'elles peuvent fournir. Cependant, dans le cas des EDS, il est impératif de garder à l'esprit que additionnellement aux erreurs numériques qui peuvent être constatées grâce, dans le cadre de ces travaux, aux solutions analytiques, il existe une erreur de modèle due aux configurations choisies. En effet dans le cas des schémas de discrétisation, les coefficients de dérive et de diffusion des processus doivent être eux-même estimés dans les cas réels et par conséquent ne sont pas parfaits. 
Ces "simplifications" entrainent que même la solution analytique (lorsqu'elle peut être calculée selon nos modèles) ne reflète pas exactement la valeur de l'objet recherchée. Nos estimateurs reste cependant les meilleurs approximations que nous puissions exploiter, en utilisant bien sûr les schémas de discrétisation et méthodes les plus performantes à notre disposition. 
  
\clearpage
\newpage
\addcontentsline{toc}{section}{\numberline{}{Références}}
\begin{thebibliography}{9}
\bibitem{Lamberton} Damien Lamberton, Bernard Lapeyre {\em Introduction au Calcul Stochastique Appliqué à la Finance}, Ellipses, $3^{\text{ème}}$ édtion, 2012.
\bibitem{MCarlo} Carl Graham , Denis Talay, {\em Simulation stochastique et méthodes de Monte-Carlo}, Ecole polytechnique, 2011.


\end{thebibliography}




\end{document}
